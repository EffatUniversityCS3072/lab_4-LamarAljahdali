% Options for packages loaded elsewhere
\PassOptionsToPackage{unicode}{hyperref}
\PassOptionsToPackage{hyphens}{url}
%
\documentclass[
]{article}
\usepackage{amsmath,amssymb}
\usepackage{lmodern}
\usepackage{iftex}
\ifPDFTeX
  \usepackage[T1]{fontenc}
  \usepackage[utf8]{inputenc}
  \usepackage{textcomp} % provide euro and other symbols
\else % if luatex or xetex
  \usepackage{unicode-math}
  \defaultfontfeatures{Scale=MatchLowercase}
  \defaultfontfeatures[\rmfamily]{Ligatures=TeX,Scale=1}
\fi
% Use upquote if available, for straight quotes in verbatim environments
\IfFileExists{upquote.sty}{\usepackage{upquote}}{}
\IfFileExists{microtype.sty}{% use microtype if available
  \usepackage[]{microtype}
  \UseMicrotypeSet[protrusion]{basicmath} % disable protrusion for tt fonts
}{}
\makeatletter
\@ifundefined{KOMAClassName}{% if non-KOMA class
  \IfFileExists{parskip.sty}{%
    \usepackage{parskip}
  }{% else
    \setlength{\parindent}{0pt}
    \setlength{\parskip}{6pt plus 2pt minus 1pt}}
}{% if KOMA class
  \KOMAoptions{parskip=half}}
\makeatother
\usepackage{xcolor}
\usepackage[margin=1in]{geometry}
\usepackage{color}
\usepackage{fancyvrb}
\newcommand{\VerbBar}{|}
\newcommand{\VERB}{\Verb[commandchars=\\\{\}]}
\DefineVerbatimEnvironment{Highlighting}{Verbatim}{commandchars=\\\{\}}
% Add ',fontsize=\small' for more characters per line
\usepackage{framed}
\definecolor{shadecolor}{RGB}{248,248,248}
\newenvironment{Shaded}{\begin{snugshade}}{\end{snugshade}}
\newcommand{\AlertTok}[1]{\textcolor[rgb]{0.94,0.16,0.16}{#1}}
\newcommand{\AnnotationTok}[1]{\textcolor[rgb]{0.56,0.35,0.01}{\textbf{\textit{#1}}}}
\newcommand{\AttributeTok}[1]{\textcolor[rgb]{0.77,0.63,0.00}{#1}}
\newcommand{\BaseNTok}[1]{\textcolor[rgb]{0.00,0.00,0.81}{#1}}
\newcommand{\BuiltInTok}[1]{#1}
\newcommand{\CharTok}[1]{\textcolor[rgb]{0.31,0.60,0.02}{#1}}
\newcommand{\CommentTok}[1]{\textcolor[rgb]{0.56,0.35,0.01}{\textit{#1}}}
\newcommand{\CommentVarTok}[1]{\textcolor[rgb]{0.56,0.35,0.01}{\textbf{\textit{#1}}}}
\newcommand{\ConstantTok}[1]{\textcolor[rgb]{0.00,0.00,0.00}{#1}}
\newcommand{\ControlFlowTok}[1]{\textcolor[rgb]{0.13,0.29,0.53}{\textbf{#1}}}
\newcommand{\DataTypeTok}[1]{\textcolor[rgb]{0.13,0.29,0.53}{#1}}
\newcommand{\DecValTok}[1]{\textcolor[rgb]{0.00,0.00,0.81}{#1}}
\newcommand{\DocumentationTok}[1]{\textcolor[rgb]{0.56,0.35,0.01}{\textbf{\textit{#1}}}}
\newcommand{\ErrorTok}[1]{\textcolor[rgb]{0.64,0.00,0.00}{\textbf{#1}}}
\newcommand{\ExtensionTok}[1]{#1}
\newcommand{\FloatTok}[1]{\textcolor[rgb]{0.00,0.00,0.81}{#1}}
\newcommand{\FunctionTok}[1]{\textcolor[rgb]{0.00,0.00,0.00}{#1}}
\newcommand{\ImportTok}[1]{#1}
\newcommand{\InformationTok}[1]{\textcolor[rgb]{0.56,0.35,0.01}{\textbf{\textit{#1}}}}
\newcommand{\KeywordTok}[1]{\textcolor[rgb]{0.13,0.29,0.53}{\textbf{#1}}}
\newcommand{\NormalTok}[1]{#1}
\newcommand{\OperatorTok}[1]{\textcolor[rgb]{0.81,0.36,0.00}{\textbf{#1}}}
\newcommand{\OtherTok}[1]{\textcolor[rgb]{0.56,0.35,0.01}{#1}}
\newcommand{\PreprocessorTok}[1]{\textcolor[rgb]{0.56,0.35,0.01}{\textit{#1}}}
\newcommand{\RegionMarkerTok}[1]{#1}
\newcommand{\SpecialCharTok}[1]{\textcolor[rgb]{0.00,0.00,0.00}{#1}}
\newcommand{\SpecialStringTok}[1]{\textcolor[rgb]{0.31,0.60,0.02}{#1}}
\newcommand{\StringTok}[1]{\textcolor[rgb]{0.31,0.60,0.02}{#1}}
\newcommand{\VariableTok}[1]{\textcolor[rgb]{0.00,0.00,0.00}{#1}}
\newcommand{\VerbatimStringTok}[1]{\textcolor[rgb]{0.31,0.60,0.02}{#1}}
\newcommand{\WarningTok}[1]{\textcolor[rgb]{0.56,0.35,0.01}{\textbf{\textit{#1}}}}
\usepackage{longtable,booktabs,array}
\usepackage{calc} % for calculating minipage widths
% Correct order of tables after \paragraph or \subparagraph
\usepackage{etoolbox}
\makeatletter
\patchcmd\longtable{\par}{\if@noskipsec\mbox{}\fi\par}{}{}
\makeatother
% Allow footnotes in longtable head/foot
\IfFileExists{footnotehyper.sty}{\usepackage{footnotehyper}}{\usepackage{footnote}}
\makesavenoteenv{longtable}
\usepackage{graphicx}
\makeatletter
\def\maxwidth{\ifdim\Gin@nat@width>\linewidth\linewidth\else\Gin@nat@width\fi}
\def\maxheight{\ifdim\Gin@nat@height>\textheight\textheight\else\Gin@nat@height\fi}
\makeatother
% Scale images if necessary, so that they will not overflow the page
% margins by default, and it is still possible to overwrite the defaults
% using explicit options in \includegraphics[width, height, ...]{}
\setkeys{Gin}{width=\maxwidth,height=\maxheight,keepaspectratio}
% Set default figure placement to htbp
\makeatletter
\def\fps@figure{htbp}
\makeatother
\setlength{\emergencystretch}{3em} % prevent overfull lines
\providecommand{\tightlist}{%
  \setlength{\itemsep}{0pt}\setlength{\parskip}{0pt}}
\setcounter{secnumdepth}{-\maxdimen} % remove section numbering
\ifLuaTeX
  \usepackage{selnolig}  % disable illegal ligatures
\fi
\IfFileExists{bookmark.sty}{\usepackage{bookmark}}{\usepackage{hyperref}}
\IfFileExists{xurl.sty}{\usepackage{xurl}}{} % add URL line breaks if available
\urlstyle{same} % disable monospaced font for URLs
\hypersetup{
  hidelinks,
  pdfcreator={LaTeX via pandoc}}

\author{}
\date{\vspace{-2.5em}}

\begin{document}

title: ``CARET\_Lab'' author:``Lamar Aljahdali'' output: html\_document
date: ``2023-05-24'' ---

Package loading

\begin{Shaded}
\begin{Highlighting}[]
\FunctionTok{library}\NormalTok{(caret)}
\end{Highlighting}
\end{Shaded}

\begin{verbatim}
## Warning: package 'caret' was built under R version 4.2.3
\end{verbatim}

\begin{verbatim}
## Loading required package: ggplot2
\end{verbatim}

\begin{verbatim}
## Warning: package 'ggplot2' was built under R version 4.2.3
\end{verbatim}

\begin{verbatim}
## Loading required package: lattice
\end{verbatim}

\begin{Shaded}
\begin{Highlighting}[]
\FunctionTok{library}\NormalTok{(lattice)}
\end{Highlighting}
\end{Shaded}

Load Data

\begin{Shaded}
\begin{Highlighting}[]
\CommentTok{\# attach the iris dataset to the environment}
\FunctionTok{data}\NormalTok{(iris)}
\CommentTok{\# rename the dataset}
\NormalTok{dataset }\OtherTok{\textless{}{-}}\NormalTok{ iris}
\end{Highlighting}
\end{Shaded}

Task1: Create a Validation/Training Dataset You need to split the loaded
dataset into two, 80\% of which we will use to train our models and 20\%
that we will hold back as a validation dataset. Hint: use
createDataPartition function

\begin{Shaded}
\begin{Highlighting}[]
\CommentTok{\# Create the training and test datasets}
\FunctionTok{set.seed}\NormalTok{(}\DecValTok{100}\NormalTok{)}
\CommentTok{\# Step 1: Get row numbers for the training data}
\NormalTok{trainRowNumbers }\OtherTok{\textless{}{-}} \FunctionTok{createDataPartition}\NormalTok{(dataset}\SpecialCharTok{$}\NormalTok{Species, }\AttributeTok{p=}\FloatTok{0.8}\NormalTok{, }\AttributeTok{list=}\ConstantTok{FALSE}\NormalTok{)}
\CommentTok{\# Step 2: Create the training dataset}
\NormalTok{trainData }\OtherTok{\textless{}{-}}\NormalTok{ dataset[trainRowNumbers,]}
\CommentTok{\# Step 3: Create the test dataset}
\NormalTok{testData }\OtherTok{\textless{}{-}}\NormalTok{ dataset[}\SpecialCharTok{{-}}\NormalTok{trainRowNumbers,]}
\end{Highlighting}
\end{Shaded}

Task2: Summarize Dataset Use skimr library to summarize the dataset

\begin{Shaded}
\begin{Highlighting}[]
\FunctionTok{library}\NormalTok{(skimr)}
\end{Highlighting}
\end{Shaded}

\begin{verbatim}
## Warning: package 'skimr' was built under R version 4.2.3
\end{verbatim}

\begin{Shaded}
\begin{Highlighting}[]
\NormalTok{skimmed}\OtherTok{\textless{}{-}}\FunctionTok{skim\_to\_wide}\NormalTok{(trainData)}
\end{Highlighting}
\end{Shaded}

\begin{verbatim}
## Warning: 'skim_to_wide' is deprecated.
## Use 'skim()' instead.
## See help("Deprecated")
\end{verbatim}

\begin{Shaded}
\begin{Highlighting}[]
\DocumentationTok{\#\# Warning: \textquotesingle{}skim\_to\_wide\textquotesingle{} is deprecated.}
\DocumentationTok{\#\# Use \textquotesingle{}skim()\textquotesingle{} instead.}
\DocumentationTok{\#\# See help("Deprecated")}
\NormalTok{skimmed}
\end{Highlighting}
\end{Shaded}

\begin{longtable}[]{@{}ll@{}}
\caption{Data summary}\tabularnewline
\toprule()
\endhead
Name & Piped data \\
Number of rows & 120 \\
Number of columns & 5 \\
\_\_\_\_\_\_\_\_\_\_\_\_\_\_\_\_\_\_\_\_\_\_\_ & \\
Column type frequency: & \\
factor & 1 \\
numeric & 4 \\
\_\_\_\_\_\_\_\_\_\_\_\_\_\_\_\_\_\_\_\_\_\_\_\_ & \\
Group variables & None \\
\bottomrule()
\end{longtable}

\textbf{Variable type: factor}

\begin{longtable}[]{@{}
  >{\raggedright\arraybackslash}p{(\columnwidth - 10\tabcolsep) * \real{0.1728}}
  >{\raggedleft\arraybackslash}p{(\columnwidth - 10\tabcolsep) * \real{0.1235}}
  >{\raggedleft\arraybackslash}p{(\columnwidth - 10\tabcolsep) * \real{0.1728}}
  >{\raggedright\arraybackslash}p{(\columnwidth - 10\tabcolsep) * \real{0.0988}}
  >{\raggedleft\arraybackslash}p{(\columnwidth - 10\tabcolsep) * \real{0.1111}}
  >{\raggedright\arraybackslash}p{(\columnwidth - 10\tabcolsep) * \real{0.3210}}@{}}
\toprule()
\begin{minipage}[b]{\linewidth}\raggedright
skim\_variable
\end{minipage} & \begin{minipage}[b]{\linewidth}\raggedleft
n\_missing
\end{minipage} & \begin{minipage}[b]{\linewidth}\raggedleft
complete\_rate
\end{minipage} & \begin{minipage}[b]{\linewidth}\raggedright
ordered
\end{minipage} & \begin{minipage}[b]{\linewidth}\raggedleft
n\_unique
\end{minipage} & \begin{minipage}[b]{\linewidth}\raggedright
top\_counts
\end{minipage} \\
\midrule()
\endhead
Species & 0 & 1 & FALSE & 3 & set: 40, ver: 40, vir: 40 \\
\bottomrule()
\end{longtable}

\textbf{Variable type: numeric}

\begin{longtable}[]{@{}
  >{\raggedright\arraybackslash}p{(\columnwidth - 20\tabcolsep) * \real{0.1867}}
  >{\raggedleft\arraybackslash}p{(\columnwidth - 20\tabcolsep) * \real{0.1333}}
  >{\raggedleft\arraybackslash}p{(\columnwidth - 20\tabcolsep) * \real{0.1867}}
  >{\raggedleft\arraybackslash}p{(\columnwidth - 20\tabcolsep) * \real{0.0667}}
  >{\raggedleft\arraybackslash}p{(\columnwidth - 20\tabcolsep) * \real{0.0667}}
  >{\raggedleft\arraybackslash}p{(\columnwidth - 20\tabcolsep) * \real{0.0533}}
  >{\raggedleft\arraybackslash}p{(\columnwidth - 20\tabcolsep) * \real{0.0533}}
  >{\raggedleft\arraybackslash}p{(\columnwidth - 20\tabcolsep) * \real{0.0533}}
  >{\raggedleft\arraybackslash}p{(\columnwidth - 20\tabcolsep) * \real{0.0533}}
  >{\raggedleft\arraybackslash}p{(\columnwidth - 20\tabcolsep) * \real{0.0667}}
  >{\raggedright\arraybackslash}p{(\columnwidth - 20\tabcolsep) * \real{0.0800}}@{}}
\toprule()
\begin{minipage}[b]{\linewidth}\raggedright
skim\_variable
\end{minipage} & \begin{minipage}[b]{\linewidth}\raggedleft
n\_missing
\end{minipage} & \begin{minipage}[b]{\linewidth}\raggedleft
complete\_rate
\end{minipage} & \begin{minipage}[b]{\linewidth}\raggedleft
mean
\end{minipage} & \begin{minipage}[b]{\linewidth}\raggedleft
sd
\end{minipage} & \begin{minipage}[b]{\linewidth}\raggedleft
p0
\end{minipage} & \begin{minipage}[b]{\linewidth}\raggedleft
p25
\end{minipage} & \begin{minipage}[b]{\linewidth}\raggedleft
p50
\end{minipage} & \begin{minipage}[b]{\linewidth}\raggedleft
p75
\end{minipage} & \begin{minipage}[b]{\linewidth}\raggedleft
p100
\end{minipage} & \begin{minipage}[b]{\linewidth}\raggedright
hist
\end{minipage} \\
\midrule()
\endhead
Sepal.Length & 0 & 1 & 5.86 & 0.82 & 4.3 & 5.1 & 5.8 & 6.4 & 7.7 &
▃▇▇▅▂ \\
Sepal.Width & 0 & 1 & 3.07 & 0.43 & 2.2 & 2.8 & 3.0 & 3.4 & 4.4 &
▃▇▇▂▁ \\
Petal.Length & 0 & 1 & 3.77 & 1.78 & 1.0 & 1.6 & 4.4 & 5.1 & 6.9 &
▇▁▅▇▂ \\
Petal.Width & 0 & 1 & 1.20 & 0.77 & 0.1 & 0.3 & 1.3 & 1.8 & 2.5 &
▇▁▇▅▃ \\
\bottomrule()
\end{longtable}

Task3: split input and output It is the time to seperate the input
attributes and the output attributes. call the inputs attributes x and
the output attribute (or class) y.

\begin{Shaded}
\begin{Highlighting}[]
\NormalTok{x }\OtherTok{=}\NormalTok{ trainData[,}\DecValTok{1}\SpecialCharTok{:}\DecValTok{4}\NormalTok{]}
\NormalTok{y }\OtherTok{=}\NormalTok{ trainData[,}\DecValTok{5}\NormalTok{]}
\end{Highlighting}
\end{Shaded}

Task4: Train Control for Validation Test

We will use 10-fold crossvalidation to estimate accuracy.

\begin{Shaded}
\begin{Highlighting}[]
\CommentTok{\# Run algorithms using 10{-}fold cross validation}
\NormalTok{control }\OtherTok{\textless{}{-}} \FunctionTok{trainControl}\NormalTok{(}\AttributeTok{method=}\StringTok{"cv"}\NormalTok{, }\AttributeTok{number=}\DecValTok{10}\NormalTok{)}
\NormalTok{metric }\OtherTok{\textless{}{-}} \StringTok{"Accuracy"}
\end{Highlighting}
\end{Shaded}

Task5: Model Training Train 5 different algorithms using `train'
function:

\begin{itemize}
\tightlist
\item
  Linear Discriminant Analysis (LDA)
\item
  Classification and Regression Trees (CART).
\item
  k-Nearest Neighbors (kNN).
\item
  Support Vector Machines (SVM) with a linear kernel.
\item
  Random Forest (RF)
\end{itemize}

\begin{Shaded}
\begin{Highlighting}[]
\CommentTok{\# Set the seed for reproducibility}
\FunctionTok{set.seed}\NormalTok{(}\DecValTok{100}\NormalTok{)}
\CommentTok{\# Train the model using LDA on the training data itself.}
\NormalTok{lda\_model }\OtherTok{=} \FunctionTok{train}\NormalTok{(Species }\SpecialCharTok{\textasciitilde{}}\NormalTok{ ., }\AttributeTok{data=}\NormalTok{trainData, }\AttributeTok{method=}\StringTok{\textquotesingle{}lda\textquotesingle{}}\NormalTok{, }
\AttributeTok{trControl=}\NormalTok{control, }\AttributeTok{metric=}\NormalTok{metric)}
\CommentTok{\#{-}{-}{-}{-}{-}{-}{-}{-}{-}{-}{-}{-}{-}{-}{-}{-}{-}{-}{-}{-}{-}{-}{-}{-}{-}{-}{-}{-}{-}{-}{-}{-}{-}{-}{-}{-}{-}{-}{-}{-}{-}{-}{-}{-}{-}{-}{-}{-}{-}{-}{-}{-}{-}{-}{-}{-}{-}{-}{-}{-}{-}{-}{-}{-}{-}{-}{-}{-}{-}{-}{-}{-}{-}{-}{-}}
\CommentTok{\# Set the seed for reproducibility}
\FunctionTok{set.seed}\NormalTok{(}\DecValTok{100}\NormalTok{)}
\CommentTok{\# Train the model using rpart on the training data itself.}
\NormalTok{rpart\_model }\OtherTok{=} \FunctionTok{train}\NormalTok{(Species }\SpecialCharTok{\textasciitilde{}}\NormalTok{ ., }\AttributeTok{data=}\NormalTok{trainData, }\AttributeTok{method=}\StringTok{\textquotesingle{}rpart\textquotesingle{}}\NormalTok{, }
\AttributeTok{trControl=}\NormalTok{control, }\AttributeTok{metric=}\NormalTok{metric)}
\CommentTok{\#{-}{-}{-}{-}{-}{-}{-}{-}{-}{-}{-}{-}{-}{-}{-}{-}{-}{-}{-}{-}{-}{-}{-}{-}{-}{-}{-}{-}{-}{-}{-}{-}{-}{-}{-}{-}{-}{-}{-}{-}{-}{-}{-}{-}{-}{-}{-}{-}{-}{-}{-}{-}{-}{-}{-}{-}{-}{-}{-}{-}{-}{-}{-}{-}{-}{-}{-}{-}{-}{-}{-}{-}{-}{-}{-}}
\CommentTok{\# Set the seed for reproducibility}
\FunctionTok{set.seed}\NormalTok{(}\DecValTok{100}\NormalTok{)}
\CommentTok{\# Train the model using kNN on the training data itself.}
\NormalTok{knn\_model }\OtherTok{=} \FunctionTok{train}\NormalTok{(Species }\SpecialCharTok{\textasciitilde{}}\NormalTok{ ., }\AttributeTok{data=}\NormalTok{trainData, }\AttributeTok{method=}\StringTok{\textquotesingle{}knn\textquotesingle{}}\NormalTok{, }
\AttributeTok{trControl=}\NormalTok{control, }\AttributeTok{metric=}\NormalTok{metric)}
\CommentTok{\#{-}{-}{-}{-}{-}{-}{-}{-}{-}{-}{-}{-}{-}{-}{-}{-}{-}{-}{-}{-}{-}{-}{-}{-}{-}{-}{-}{-}{-}{-}{-}{-}{-}{-}{-}{-}{-}{-}{-}{-}{-}{-}{-}{-}{-}{-}{-}{-}{-}{-}{-}{-}{-}{-}{-}{-}{-}{-}{-}{-}{-}{-}{-}{-}{-}{-}{-}{-}{-}{-}{-}{-}{-}{-}{-}}
\CommentTok{\# Set the seed for reproducibility}
\FunctionTok{set.seed}\NormalTok{(}\DecValTok{100}\NormalTok{)}
\CommentTok{\# Train the model using svmRadial on the training data itself.}
\NormalTok{svm\_model }\OtherTok{=} \FunctionTok{train}\NormalTok{(Species }\SpecialCharTok{\textasciitilde{}}\NormalTok{ ., }\AttributeTok{data=}\NormalTok{trainData, }\AttributeTok{method=}\StringTok{\textquotesingle{}svmRadial\textquotesingle{}}\NormalTok{, }
\AttributeTok{trControl=}\NormalTok{control, }\AttributeTok{metric=}\NormalTok{metric)}
\CommentTok{\#{-}{-}{-}{-}{-}{-}{-}{-}{-}{-}{-}{-}{-}{-}{-}{-}{-}{-}{-}{-}{-}{-}{-}{-}{-}{-}{-}{-}{-}{-}{-}{-}{-}{-}{-}{-}{-}{-}{-}{-}{-}{-}{-}{-}{-}{-}{-}{-}{-}{-}{-}{-}{-}{-}{-}{-}{-}{-}{-}{-}{-}{-}{-}{-}{-}{-}{-}{-}{-}{-}{-}{-}{-}{-}{-}}
\CommentTok{\# Set the seed for reproducibility}
\FunctionTok{set.seed}\NormalTok{(}\DecValTok{100}\NormalTok{)}
\CommentTok{\# Train the model using rf on the training data itself.}
\NormalTok{rf\_model }\OtherTok{=} \FunctionTok{train}\NormalTok{(Species }\SpecialCharTok{\textasciitilde{}}\NormalTok{ ., }\AttributeTok{data=}\NormalTok{trainData, }\AttributeTok{method=}\StringTok{\textquotesingle{}rf\textquotesingle{}}\NormalTok{, }\AttributeTok{trControl=}\NormalTok{control, }
\AttributeTok{metric=}\NormalTok{metric)}
\end{Highlighting}
\end{Shaded}

Task6: Select the Best Model We now have 5 models and accuracy
estimations for each. We need to compare the models to each other and
select the most accurate. Use resamples function to complete this task

\begin{Shaded}
\begin{Highlighting}[]
\CommentTok{\# Compare model performances using resample()}
\NormalTok{models\_compare}\OtherTok{\textless{}{-}}\FunctionTok{resamples}\NormalTok{(}\FunctionTok{list}\NormalTok{(}\AttributeTok{lda=}\NormalTok{lda\_model, }\AttributeTok{rpart=}\NormalTok{rpart\_model, }
\AttributeTok{knn=}\NormalTok{knn\_model, }\AttributeTok{svm=}\NormalTok{svm\_model, }\AttributeTok{rf=}\NormalTok{rf\_model))}
\CommentTok{\# Summary of the models performances}
\FunctionTok{summary}\NormalTok{(models\_compare)}
\end{Highlighting}
\end{Shaded}

\begin{verbatim}
## 
## Call:
## summary.resamples(object = models_compare)
## 
## Models: lda, rpart, knn, svm, rf 
## Number of resamples: 10 
## 
## Accuracy 
##            Min.   1st Qu.    Median      Mean 3rd Qu. Max. NA's
## lda   0.9166667 0.9375000 1.0000000 0.9750000       1    1    0
## rpart 0.8333333 0.8541667 0.9166667 0.9250000       1    1    0
## knn   0.9166667 1.0000000 1.0000000 0.9833333       1    1    0
## svm   0.8333333 0.9166667 0.9583333 0.9416667       1    1    0
## rf    0.9166667 0.9166667 1.0000000 0.9666667       1    1    0
## 
## Kappa 
##        Min. 1st Qu. Median   Mean 3rd Qu. Max. NA's
## lda   0.875 0.90625 1.0000 0.9625       1    1    0
## rpart 0.750 0.78125 0.8750 0.8875       1    1    0
## knn   0.875 1.00000 1.0000 0.9750       1    1    0
## svm   0.750 0.87500 0.9375 0.9125       1    1    0
## rf    0.875 0.87500 1.0000 0.9500       1    1    0
\end{verbatim}

What was the most accurate model?

Task7: Make Prediction (Confusion Matrix) Now we want to get an idea of
the accuracy of the best model on our validation set. Use `predict' and
confusionMatrix functions to complete this task.

\begin{Shaded}
\begin{Highlighting}[]
\CommentTok{\# Predict on testData using lda\_model}
\NormalTok{predicted }\OtherTok{\textless{}{-}} \FunctionTok{predict}\NormalTok{(lda\_model, testData)}
\FunctionTok{confusionMatrix}\NormalTok{(}\AttributeTok{reference=}\NormalTok{ testData}\SpecialCharTok{$}\NormalTok{Species,}\AttributeTok{data =}\NormalTok{ predicted)}
\end{Highlighting}
\end{Shaded}

\begin{verbatim}
## Confusion Matrix and Statistics
## 
##             Reference
## Prediction   setosa versicolor virginica
##   setosa         10          0         0
##   versicolor      0          9         0
##   virginica       0          1        10
## 
## Overall Statistics
##                                           
##                Accuracy : 0.9667          
##                  95% CI : (0.8278, 0.9992)
##     No Information Rate : 0.3333          
##     P-Value [Acc > NIR] : 2.963e-13       
##                                           
##                   Kappa : 0.95            
##                                           
##  Mcnemar's Test P-Value : NA              
## 
## Statistics by Class:
## 
##                      Class: setosa Class: versicolor Class: virginica
## Sensitivity                 1.0000            0.9000           1.0000
## Specificity                 1.0000            1.0000           0.9500
## Pos Pred Value              1.0000            1.0000           0.9091
## Neg Pred Value              1.0000            0.9524           1.0000
## Prevalence                  0.3333            0.3333           0.3333
## Detection Rate              0.3333            0.3000           0.3333
## Detection Prevalence        0.3333            0.3000           0.3667
## Balanced Accuracy           1.0000            0.9500           0.9750
\end{verbatim}

\end{document}
